%%%%%%%%%%%%%%%%%
% This is an sample CV template created using altacv.cls
% (v1.1.2, 1 February 2017) written by LianTze Lim (liantze@gmail.com). Now compiles with pdfLaTeX, XeLaTeX and LuaLaTeX.
%
%% It may be distributed and/or modified under the
%% conditions of the LaTeX Project Public License, either version 1.3
%% of this license or (at your option) any later version.
%% The latest version of this license is in
%%    http://www.latex-project.org/lppl.txt
%% and version 1.3 or later is part of all distributions of LaTeX
%% version 2003/12/01 or later.
%%%%%%%%%%%%%%%%

%% If you need to pass whatever options to xcolor
\PassOptionsToPackage{dvipsnames}{xcolor}

%% If you are using \orcid or academicons
%% icons, make sure you have the academicons
%% option here, and compile with XeLaTeX
%% or LuaLaTeX.
% \documentclass[10pt,a4paper,academicons]{altacv}
\documentclass[10pt,a4paper]{altacv}

%% AltaCV uses the fontawesome and academicon fonts
%% and packages.
%% See texdoc.net/pkg/fontawecome and http://texdoc.net/pkg/academicons for full list of symbols.
%%
%% Compile with LuaLaTeX for best results. If you
%% want to use XeLaTeX, you may need to install
%% Academicons.ttf in your operating system's font
%% folder.


% Change the page layout if you need to
\geometry{left=1cm,right=9cm,marginparwidth=6.8cm,marginparsep=1.2cm,top=1.25cm,bottom=1.25cm}

% Change the font if you want to.

% If using pdflatex:
% \usepackage[utf8]{inputenc}
% \usepackage[T1]{fontenc}
% \usepackage[default]{lato}
\usepackage[english,russian]{babel}   %% загружает пакет многоязыковой вёрстки
\usepackage{fontspec} 
\usepackage{setspace}

% If using xelatex or lualatex:
\setmainfont{Lato}

% Change the colours if you want to
\definecolor{Mulberry}{HTML}{6600cc}
\definecolor{SlateGrey}{HTML}{29294F}
\definecolor{LightGrey}{HTML}{666666}
\definecolor{MyHead}{HTML}{00005c}
\colorlet{heading}{MyHead}
\colorlet{accent}{Mulberry}
\colorlet{emphasis}{SlateGrey}
\colorlet{body}{LightGrey}

% Change the bullets for itemize and rating marker
% for \cvskill if you want to
\renewcommand{\itemmarker}{{\small\textbullet}}
\renewcommand{\ratingmarker}{\faCircle}

%% sample.bib contains your publications
\addbibresource{sample.bib}

\begin{document}
\name{Антон Лиознов}
\tagline{магистрант}
\photo{2.8cm}{me}
\personalinfo{%
  % Not all of these are required!
  % You can add your own with \printinfo{symbol}{detail}
  \email{Lavton@gmail.cpm}
  \phone{+7(904)635-52-51}
  \location{Санкт-Петербург, Россия}
  \linkedin{linkedin.com/in/lavton/}
  \github{github.com/Lavton}
  \vk{vk.com/lavton}
  

  %% You MUST add the academicons option to \documentclass, then compile with LuaLaTeX or XeLaTeX, if you want to use \orcid or other academicons commands.
%   \orcid{orcid.org/0000-0000-0000-0000}
}

%% Make the header extend all the way to the right, if you want. Extend the right margin by 8cm (=6.8cm marginparwidth + 1.2cm marginparsep)
\begin{adjustwidth}{}{-8cm}
\makecvheader
\end{adjustwidth}

%% Provide the file name containing the sidebar contents as an optional parameter to \cvsection.
%% You can always just use \marginpar{...} if you do
%% not need to align the top of the contents to any
%% \cvsection title in the "main" bar.
\cvsection[sample-p1sidebar.tex]{Опыт работы}

\cvevent{программист-стажёр}{OKKO-фильмы}{лето 2014}{}
\begin{itemize}
\item Оптимизировал работу с внешними источниками
\item Добился ускорения работы на порядок
\item Улучшил качество поиска
\end{itemize}
\cvtag{Java} \cvtag{Grails} \cvtag{Postgresql} \cvtag{Git}
\divider

\cvevent{Тестировщик задач по физике}{Competentum}{Лето 2014}{}
\begin{itemize}
\item Искал ошибки в условиях и решениях
\item Тестировал работу с сайтом с точки зрения пользователя
\end{itemize}
\cvtag{black box testing}
\divider

\cvevent{Преподаватель физики}{Формула Единства}{лето 2015, зима 2016}{}
\begin{itemize}
\item Сочетал различные способы объяснения материала
\item Полностью готовил программу
\end{itemize}

\cvsection{Проекты}

\cvevent{Компьютерное моделирование ускорения космических лучей на фронтах волны сверхновой}{Политехнический Университет}{2014 -- 2015}{}
\begin{itemize}
\item Построение разностной схемы и сравнение со стахастическими подходами
\end{itemize}
\cvtag{C++} \cvtag{Git} \cvtag{gle}
\divider

\cvevent{Онлайн-курс "Основы программирования для Linux"}{Stepik}{2016}{}
\begin{itemize}
\item Работал над серверной частью системы проверки задач и RESTful API
\end{itemize}
\cvtag{Python} \cvtag{Docker}
\divider

\cvevent{Общее руководство}{Вожатник}{2016}{}
\begin{itemize}
\item Создал сайт и системы автоматизации
\item Занимаюсь сбором статистики
\end{itemize}
\cvtag{PHP} \cvtag{YII2} \cvtag{JS} \cvtag{mySQL} \cvtag{Python} \cvtag{R}

\medskip

%\cvsection{A Day of My Life}

% Adapted from @Jake's answer from http://tex.stackexchange.com/a/82729/226
% \wheelchart{outer radius}{inner radius}{
% comma-separated list of value/text width/color/detail}
%\wheelchart{1.5cm}{0.5cm}{%
  %6/8em/accent!30/{Sleep,\\beautiful sleep},
  %3/8em/accent!40/Hopeful novelist by night,
%   8/8em/accent!60/Daytime job,
%   2/10em/accent/Sports and relaxation,
%   5/6em/accent!20/Spending time with family
% }

\clearpage
\cvsection[page2sidebar]{Publications}

\nocite{*}

\printbibliography[heading=pubtype,title={\printinfo{\faBook}{Books}},type=book]

\divider

\printbibliography[heading=pubtype,title={\printinfo{\faFileTextO}{Journal Articles}},type=article]

\divider

\printbibliography[heading=pubtype,title={\printinfo{\faGroup}{Conference Proceedings}},type=inproceedings]

%% If the NEXT page doesn't start with a \cvsection but you'd
%% still like to add a sidebar, then use this command on THIS
%% page to add it. The optional argument lets you pull up the
%% sidebar a bit so that it looks aligned with the top of the
%% main column.
% \addnextpagesidebar[-1ex]{page3sidebar}


\end{document}
